\documentclass{article}
\usepackage{ws_template}

\usepackage{amsmath}
\usepackage{listings}

\title{homework sheet 01}
\author{
	\name{Denys Sobchyshak}\\
	\imat{03636581}\\
	\email{denys.sobchyshak@gmail.com}
	\And
	\name{Serge Zakharov} \\
	\imat{03636642}\\
	\email{zsn2008@gmail.com}
}

\begin{document}
	\maketitle
	
	\section{Problem 1: computeing eigenvalues and eigenvectors of a given matrix}
	To find eigenvalues of the matrix A we have to solve $ det(A-\lambda I)=0 $. \\
	$
	det(A-\lambda I)= det
	\begin{pmatrix} 
		2-\lambda & -1 & 0  \\ 
		-1 & 2-\lambda & -1 \\ 
		0 & -1 & 2-\lambda
	\end{pmatrix}\\
	=(2-\lambda)^3-(2-\lambda)(-1)(-1)-(2-\lambda)(-1)(-1)\\
	=(2-\lambda)^3-2(2-\lambda)=(2-\lambda)((2-\lambda)^2-2)=0
	$\\
	Thus we can see that $\lambda_1=2, \lambda_{2,3}=2\mp\sqrt{2}$. To find the corresponding eigenvectors we substitute our eigenvalues into the eigenvalue equation.\\
	\begin{enumerate}
		\item For $\lambda=2$:\\
		$
		A-2I=0 \Rightarrow
		\left(\begin{array}{ccc|c}
		2-2 & -1 & 0 & 0 \\ 
		-1 & 2-2 & -1 & 0\\ 
		0 & -1 & 2-2 & 0
		\end{array}\right)
		\Rightarrow \left\{
		\begin{array}{cc}
		x=-z\\
		y=0\\
		\end{array}\right.
		$\\
		Thus using $z=1$ we get the next basis eigenvector $v_1=\begin{pmatrix}1 \\0\\ -1\end{pmatrix}$.

		\item For $\lambda=2-\sqrt{2}$:\\
		$
		A-2I=0 \Rightarrow
		\left(\begin{array}{ccc|c}
		2-2+\sqrt{2} & -1 & 0 & 0 \\
		0 & -1 & 2-2+\sqrt{2} & 0 \\
		-1 & 2-2+\sqrt{2} & -1 & 0 
		\end{array}\right)
		\Rightarrow \left\{
		\begin{array}{cc}
		x=\frac{y}{\sqrt{2}}\\
		y=\sqrt{2}z\\
		z=\frac{y}{\sqrt{2}}\\
		\end{array}\right.
		$\\
		Thus using $z=1$ we get the next basis eigenvector $v_2=\begin{pmatrix}1 \\\sqrt{2}\\ 1\end{pmatrix}$.
		
		\item For $\lambda=2+\sqrt{2}$:\\
		$
		A-2I=0 \Rightarrow
		\left(\begin{array}{ccc|c}
		2-2-\sqrt{2} & -1 & 0 & 0 \\ 
		-1 & 2-2-\sqrt{2} & -1 & 0\\ 
		0 & -1 & 2-2-\sqrt{2} & 0
		\end{array}\right)
		\Rightarrow \left\{
		\begin{array}{cc}
		x=-\frac{y}{\sqrt{2}}\\
		y=-\sqrt{2}z\\
		z=-\frac{y}{\sqrt{2}}\\
		\end{array}\right.
		$\\
		Thus using $z=1$ we get the next basis eigenvector $v_3=\begin{pmatrix}1 \\-\sqrt{2} \\ 1\end{pmatrix}$.
	\end{enumerate}
	The corresponding python/numpy code looks as follows:\\
	\lstinputlisting[language=Python,firstline=21]{scripts/ws1-linear-algebra.py}
\end{document}