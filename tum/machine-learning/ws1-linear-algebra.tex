\documentclass{article}
\usepackage{ws_template}

\usepackage{amsmath}
\usepackage{listings}

\title{homework sheet 01}
\author{
	\name{Denys Sobchyshak}\\
	\imat{03636581}\\
	\email{denys.sobchyshak@gmail.com}
	\And
	\name{Serge Zakharov} \\
	\imat{03636642}\\
	\email{zsn2008@gmail.com}
}

\begin{document}
	\maketitle
	
	\section{Problem}
	To find eigenvalues of the matrix A we have to solve $ det(A-\lambda I)=0 $ which cane be derived from an eigenvelue equation. \\
	$
	det(A-\lambda I)= det
	\begin{pmatrix} 
		2-\lambda & -1 & 0  \\ 
		-1 & 2-\lambda & -1 \\ 
		0 & -1 & 2-\lambda
	\end{pmatrix}\\
	=(2-\lambda)^3-(2-\lambda)(-1)(-1)-(2-\lambda)(-1)(-1)\\
	=(2-\lambda)^3-2(2-\lambda)=(2-\lambda)((2-\lambda)^2-2)=0
	$\\
	Thus we can see that the spectrum of the matrix is given by $\lambda(A)=\left\{2, 2-\sqrt{2},2+\sqrt{2}\right\}$. To find the corresponding eigenvectors we substitute our eigenvalues into the eigenvalue equation.\\
	\begin{enumerate}
		\item For $\lambda=2$:\\
		$
		A-2I=0 \Rightarrow
		\left(\begin{array}{ccc|c}
		2-2 & -1 & 0 & 0 \\ 
		-1 & 2-2 & -1 & 0\\ 
		0 & -1 & 2-2 & 0
		\end{array}\right)
		\Rightarrow \left\{
		\begin{array}{cc}
		x=-z\\
		y=0\\
		\end{array}\right.
		$\\
		Thus using $z=1$ we get the next basis eigenvector $v_1=\begin{pmatrix}1 \\0\\ -1\end{pmatrix}$.

		\item For $\lambda=2-\sqrt{2}$:\\
		$
		A-2I=0 \Rightarrow
		\left(\begin{array}{ccc|c}
		2-2+\sqrt{2} & -1 & 0 & 0 \\
		0 & -1 & 2-2+\sqrt{2} & 0 \\
		-1 & 2-2+\sqrt{2} & -1 & 0 
		\end{array}\right)
		\Rightarrow \left\{
		\begin{array}{cc}
		x=\frac{y}{\sqrt{2}}\\
		y=\sqrt{2}z\\
		z=\frac{y}{\sqrt{2}}\\
		\end{array}\right.
		$\\
		Thus using $z=1$ we get the next basis eigenvector $v_2=\begin{pmatrix}1 \\\sqrt{2}\\ 1\end{pmatrix}$.
		
		\item For $\lambda=2+\sqrt{2}$:\\
		$
		A-2I=0 \Rightarrow
		\left(\begin{array}{ccc|c}
		2-2-\sqrt{2} & -1 & 0 & 0 \\ 
		-1 & 2-2-\sqrt{2} & -1 & 0\\ 
		0 & -1 & 2-2-\sqrt{2} & 0
		\end{array}\right)
		\Rightarrow \left\{
		\begin{array}{cc}
		x=-\frac{y}{\sqrt{2}}\\
		y=-\sqrt{2}z\\
		z=-\frac{y}{\sqrt{2}}\\
		\end{array}\right.
		$\\
		Thus using $z=1$ we get the next basis eigenvector $v_3=\begin{pmatrix}1 \\-\sqrt{2} \\ 1\end{pmatrix}$.
	\end{enumerate}
	The corresponding python/numpy code looks as follows:\\
	\lstinputlisting[language=Python,firstline=21]{scripts/ws1-linear-algebra.py}
	
	\section{Problem}
	To show that $B=UDU^{-1}$ it is sufficient to show that B is diagonalizable. A square matrix is diagonalizable if it has n distinct eigenvelues. We know that B has n linearly independent eigenvectors and by definition of linear independence all of the vectors are distinct, which implies that we have n distinct eigenvelues, since otherwise we would have less linearly independent eigenvectors. Thus matrix B is diagonalizable and according to the eigen decomposition teorem $B=UDU^{-1}$. This can also be shown as follows:\\
	$
	AP=A\begin{bmatrix} X_{1} & X_{2} & \cdots & X_n \end{bmatrix}\\
	=\begin{bmatrix} AX_{1} & AX_{2} & \cdots & AX_n \end{bmatrix}\\
	=\begin{bmatrix} \lambda_1X_{1} & \lambda_2X_{2} & \cdots & \lambda_nX_n \end{bmatrix}\\
	=\begin{bmatrix} \lambda_1X_{1} & \lambda_2X_{2} & \cdots & \lambda_nX_n \end{bmatrix}\\
	=\begin{bmatrix} 
		\lambda_1x_{1,1} & \lambda_2x_{2,1} & \cdots & \lambda_nx_{n,1} \\
		\lambda_1x_{1,2} & \lambda_2x_{2,2} & \cdots & \lambda_nx_{n,2} \\
		\vdots & \vdots & \ddots & \vdots \\
		\lambda_1x_{1,n} & \lambda_2x_{2,n} & \cdots & \lambda_nx_{n,n}
	\end{bmatrix}\\
	=\begin{bmatrix} 
	x_{1,1} & x_{2,1} & \cdots & x_{n,1} \\
	x_{1,2} & x_{2,2} & \cdots & x_{n,2} \\
	\vdots & \vdots & \ddots & \vdots \\
	x_{1,n} & x_{2,n} & \cdots & x_{n,n}
	\end{bmatrix}
	\begin{bmatrix} 
	\lambda_1 & 0 & \cdots & 0 \\
	0 & \lambda_2 & \cdots & 0 \\
	\vdots & \vdots & \ddots & \vdots \\
	0 & 0 & \cdots & \lambda_n
	\end{bmatrix}
	=PD
	$
	
	\section{Problem}
	Note: provided matrix $B$ is square.
	\begin{enumerate}
		\item Generally since any real matrix is a Hermittian matrix and the finite-dimensional spectral theorem says that any Hermitian matrix can be diagonalized by a unitary matrix, and that the resulting diagonal matrix has only real entries. This implies that all eigenvalues of a Hermitian matrix $B$ are real.\\
		
		However the proof to the first task can be obtained from eigenvelue equation $Bu=\lambda u$. If $\lambda$ is an eigenvalue of $B$, that is, $det(B-\lambda I) = 0$, then $B-\lambda I$ must be non-invertible. This means that there exist a non-zero real vector $u$ such that $Au = \lambda u$. We can always normalize $u$ so that $u^Tu = 1$. Thus, $\lambda = u^TAu$ is real. That is, the eigenvalues of a symmetric matrix are always real.
		
		\item Proof of this comes straight from the eigen decomposition theorem and is considered to be a special case where it states that any symmeteric $n \times n$ real matrix $B$ can be decomposed as $B=UDU^T$ (using definitions from problem 2).
	\end{enumerate}
	
	\section{Problem}
	\begin{enumerate}
		\item From the eigenvelue equation we know that $det(B-\lambda I)=(\lambda_1 - \lambda)(\lambda_2 - \lambda)\cdots(\lambda_n-\lambda)$ Substituting $\lambda=0$ we will obtain $det(B)=\prod_{n=1}^{n}\lambda_n.$
		\item Using the definition of a characteristic equation of a $n \times n$ matrix $B$:\\
		$p(\lambda)=det(B-\lambda I)=(-1)^n(\lambda^n-tr(B)t^{n-1}+\cdots+det(B))$\\
		and also\\
		$p(\lambda)=(-1)^n(\lambda-\lambda_1)(\lambda-\lambda_2)\cdots(\lambda-\lambda_n)$.\\
		By simply comparing coefficients we have that $tr(A)=\lambda_1+\lambda_2+\cdots+\lambda_n=\sum_{1}^{n}\lambda_n$.
	\end{enumerate}
	
	\section{Problem}
	
\end{document}